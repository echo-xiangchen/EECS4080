\documentclass[a4paper,12pt,titlepage]{article}



\usepackage{index}
\makeindex
\usepackage[a4paper, inner=1.7cm, outer=2.7cm, top=2cm, bottom=2cm, bindingoffset=1.2cm]{geometry}
\usepackage[english]{babel}

% define my own color
\usepackage{xcolor}
\definecolor{comment}{RGB}{120,120,120}
\definecolor{codenumber}{RGB}{100,100,100}
\definecolor{shade}{RGB}{240,240,240}
\definecolor{link}{RGB}{0,0,225}
\definecolor{url}{RGB}{30,144,255}

\usepackage{framed}
\usepackage{graphicx}
\usepackage{wrapfig}
\usepackage{blindtext}

%make the section title centered
\usepackage{sectsty}
\sectionfont{\centering}

\usepackage{enumitem}


% use font package
\usepackage{fontspec}
% change the font
\setmonofont{Menlo}


% provides different line thicknesses for tables
\usepackage{booktabs}

%prevents figures ever floating backwards up the current page
\usepackage{flafter}

% here for H placement parameter
\usepackage{float} 

\usepackage{makecell, caption}

% enable inserting codes
\usepackage{listings}

\lstset{
  language={[LaTeX]TeX},
  commentstyle=\color{comment},
  basicstyle=\footnotesize\ttfamily,
  numbers=left,
  numberstyle=\scriptsize\color{codenumber},
  frame=single,		% add the frame
  breaklines=true,	% sets automatic line breaking
 % texcsstyle=*\color{red},
 % identifierstyle=\color{magenta},
  extendedchars,
  escapeinside=``,	% used for escaping
  tabsize=2
}

\lstset{
  literate=*{¡\\!}{{\textcolor{red}{\textbackslash!}}}{2}
            {¡\\"}{{\textcolor{red}{\textbackslash"}}}{2}
            {¡\\\#}{{\textcolor{red}{\textbackslash\#}}}{2}
            {¡\\\$}{{\textcolor{red}{\textbackslash\$}}}{2}
            {¡\\\%}{{\textcolor{red}{\textbackslash\%}}}{2}
            {¡\\\&}{{\textcolor{red}{\textbackslash\&}}}{2}
            {¡\\'}{{\textcolor{red}{\textbackslash'}}}{2}
            {¡\\(}{{\textcolor{red}{\textbackslash(}}}{2}
            {¡\\)}{{\textcolor{red}{\textbackslash)}}}{2}
            {¡\\*}{{\textcolor{red}{\textbackslash*}}}{2}
            {¡\\+}{{\textcolor{red}{\textbackslash+}}}{2}
            {¡\\,}{{\textcolor{red}{\textbackslash,}}}{2}
            {¡\\-}{{\textcolor{red}{\textbackslash-}}}{2}
            {¡\\.}{{\textcolor{red}{\textbackslash.}}}{2}
            {¡\\/}{{\textcolor{red}{\textbackslash/}}}{2}
            {¡\\:}{{\textcolor{red}{\textbackslash:}}}{2}
            {¡\\;}{{\textcolor{red}{\textbackslash;}}}{2}
            {¡\\<}{{\textcolor{red}{\textbackslash<}}}{2}
            {¡\\=}{{\textcolor{red}{\textbackslash=}}}{2}
            {¡\\>}{{\textcolor{red}{\textbackslash>}}}{2}
            {¡\\?}{{\textcolor{red}{\textbackslash?}}}{2}
            {¡\\[}{{\textcolor{red}{\textbackslash[}}}{2}
            {¡\\\\}{{\textcolor{red}{\textbackslash\textbackslash}}}{2}
            {¡\\]}{{\textcolor{red}{\textbackslash]}}}{2}
            {¡\\\^}{{\textcolor{red}{\textbackslash\textasciicircum}}}{2}
            {¡\\\{}{{\textcolor{red}{\textbackslash\{}}}{2}
            {¡\\|}{{\textcolor{red}{\textbackslash|}}}{2}
            {¡\\\}}{{\textcolor{red}{\textbackslash\}}}}{2}
            {¡\\\~}{{\textcolor{red}{\textbackslash\textasciitilde}}}{2}
}

% E.g.
% \begin{lstlisting}
% ¡\! ¡\" ¡\# ¡\$ ¡\% ¡\& ¡\' ¡\( ¡\) ¡\* ¡\+ ¡\, ¡\- ¡\. ¡\/ ¡\: ¡\; ¡\< ¡\= ¡\> ¡\? ¡\[ ¡\\ ¡\] ¡\^ ¡\{ ¡\| ¡\} ¡\~
% \end{lstlisting}



% Use hyperlink
\usepackage{hyperref}


\urlstyle{sf}

\hypersetup{
    colorlinks=true,
    linkcolor=link,
    filecolor=magenta,      
    urlcolor=cyan,
}


% For math formulas
\usepackage{amsmath}




% documents begins here
\begin{document}

\title{\Large{\textbf{My LaTex Notes}}}
\author{By Xiang Chen (Echo)}
\date{July 18, 2019}
\maketitle
\tableofcontents
\newpage



% Basic notes section
\section{Basic Notes}
Commands start with a name [optional arguments] \{required arguments\}
\\
\\twocolumn : 2 column pages
\\titlepage : \emph{\textbf{\textbackslash maketitle|}} generates a title page
\\legno : Puts equation numbers on the left side
\\flegn : Left align equations versus center
\\twoside : Print on both sides of paper
\\openright : If twoside is used chapters begin on right hand page
\\landscape : If listed it displays in landscape
\\
\\
\rule{\linewidth}{0.1mm}






% Numbering section
\section{Page Numbering}
\begin{lstlisting}
% Use roman numeral page numbering
	\pagenumbering{roman}
\end{lstlisting}
~\\
\begin{lstlisting}
% Start numbering with page 2
	\setcounter{page}{2}
\end{lstlisting}
~\\
\\
\rule{\linewidth}{0.1mm}


% Headers and footers section
\section{Headers \& Footers}
\begin{lstlisting}
% Clear default headers & footers
	\fancyhf{}
\end{lstlisting}
~\\
\begin{lstlisting}
% Draw a decorative line at the top & bottom of the page
	\renewcommand{\headrulewidth}{2pt}
	\renewcommand{\footrulewidth}{1pt}
\end{lstlisting}
~\\
\rule{\linewidth}{0.1mm}
\\
\\Could also use different headers for even \& odd pages:
\\
\\
\begin{lstlisting}
% Leftmark : Chapter title, number, LE - left side on even pages
% Uppercase by default
	\fancyhead[LE]{\leftmark}
\end{lstlisting}
~\\
\begin{lstlisting}
% Rightmark : Chapter title, number, RO - right side on odd pages
% Make lowercase
	\fancyhead[RO]\{\nouppercase{\rightmark}
\end{lstlisting}
~\\
\begin{lstlisting}
% Use the same footer for pages
	\fancyfoot[LE,RO]{\thepage}
\end{lstlisting}
~\\
\\
\rule{\linewidth}{0.1mm}




% class section
\section{Class, Custom margins}
\begin{lstlisting}
% Document will be printed on a4 paper, using the 12pt default font
% and there will be a cover page.
	\documentclass[a4paper,12pt,titlepage]{article}
\end{lstlisting}
~\\
\\
\hspace*{10mm}\textbf{--- Paper types ---}
\\\hspace*{10mm}letterpaper (11 x 8.5 in)
\\\hspace*{10mm}a4paper (29.7 x 21 cm) -- usually use this
\\\hspace*{10mm}legalpaper (14 x 8.5 in)
\\\hspace*{10mm}a5paper (21 x 14.8 cm)
\\\hspace*{10mm}executivepaper (10.5 x 7.25 in)
\\\hspace*{10mm}b5paper (25 x 17.6 cm)
\\
\\
\hspace*{10mm}\textbf{--- Class types ---}
\\\hspace*{10mm}article -- for papers
\\\hspace*{10mm}book
\\\hspace*{10mm}letter
\\\hspace*{10mm}report
\\\hspace*{10mm}beamer --- for presentation
\\
\\
\begin{lstlisting}
% Define custom margins
	\usepackage[a4paper,inner=1.7cm,outer=2.7cm,top=2cm,bottom=2cm,
	bindingoffset=1.2cm]{geometry}
\end{lstlisting}
~\\
\\
\rule{\linewidth}{0.1mm}




% package section
\section{Packages}
You can import packages to add functionality.
\\Get more info on any package by typing \textbf{texdoc PackageName} in the terminal or command line.
\\
\\
\begin{lstlisting}
% Define that we want to use English hyphenation.
% http://mirrors.rit.edu/CTAN/macros/latex/required/babel/base/babel.pdf.
% Page 18 for list of languages.
	\usepackage[english]{babel}
\end{lstlisting}
~\\
\begin{lstlisting}
% Use Helvetica instead of the normal sans serif font
	\usepackage[scaled=.92]{helvet}
\end{lstlisting}
~\\
\hspace*{10mm}\textbf{--- Others: ---}
\\\hspace*{14mm}mathpazo (Palatino (Roman))
\\\hspace*{14mm}mathptmx (Times (Roman))
\\\hspace*{14mm}avant (Avant Garde (Sans Serif))
\\\hspace*{14mm}courier (Courier (Typewriter))
\\\hspace*{14mm}chancery (Zapf Chancery (Roman))
\\\hspace*{14mm}bookman (Bookman (Roman) Avant Garde (Sans Serif) Courier (Typewriter))
\\\hspace*{14mm}newcent (New Century, Avant Garde, Courier)
\\\hspace*{14mm}charter (Charter (Roman))
\\
\\
\\
\\
\rule{\linewidth}{0.1mm}
\\
\\Some commonly used packages:
\\
\begin{lstlisting}
% Improve justification document wide
	\usepackage{microtype}
\end{lstlisting}
~\\
\begin{lstlisting}
% Used to create filler text
	\usepackage{blindtext}
	
	E.g. \blindtext[5] --- create filler text
\end{lstlisting}
~\\
\begin{lstlisting}
% Used to include pictures
	\usepackage{graphicx}
\end{lstlisting}
~\\
\begin{lstlisting}
% Used to wrap text around pictures
	\usepackage{wrapfig}
\end{lstlisting}
~\\
\begin{lstlisting}
% Used to compact lists
	\usepackage{enumitem}
\end{lstlisting}
~\\
\begin{lstlisting}
% Improve the justification across the entire document
	\usepackage{mirotype}
\end{lstlisting}
~\\
\begin{lstlisting}
% Used to customize the page layout of your LaTeX documents
	\usepackage{fancyhdr}
\end{lstlisting}
~\\
\begin{lstlisting}
% Improve output of math formulas
	\usepackage{amsmath}
\end{lstlisting}
~\\
\begin{lstlisting}
% Automatically generate index
	\usepackage{index}
	\makeindex
\end{lstlisting}
~\\
\begin{lstlisting}
% Customize the font color
	\usepackage{xcolor}
\end{lstlisting}
~\\
\begin{lstlisting}
% Change the font background color
	\usepackage{framed}
\end{lstlisting}
~\\
\\
\rule{\linewidth}{0.1mm}







% Spacing section
\section{Spacing}
The first line after the section will not be indented.
\\
Also, multiple spaces do not matter.
\\
\begin{lstlisting}
% A built in command
	\LaTeX
\end{lstlisting}
~\\
\begin{lstlisting}
% Creates a line break
	\\
\end{lstlisting}
~\\
\begin{lstlisting}
% Prevents line breaks.
	\nolinebreak
\end{lstlisting}
~\\
\begin{lstlisting}
% Add a 10 pts of space between this line and the next.
	\\[10pt]
\end{lstlisting}
~\\
\begin{lstlisting}
% Eliminate paragraph indents.
	\usepackage{parskip}
\end{lstlisting}
~\\
\begin{lstlisting}
% Only eliminate the indent for the very next paragraph
% Every other paragraph that comes after it will continue doing indents.
	\noindent
\end{lstlisting}
~\\
\begin{lstlisting}
% Increase the line spacing : singlespacing, onehalfspacing, doublespacing
	\usepackage[onehalfspacing]{setspace}
\end{lstlisting}
~\\
\begin{lstlisting}
% Another way to add space between paragraphs.
	\bigskip
\end{lstlisting}
~\\
\\
\rule{\linewidth}{0.1mm}






% Font section
\section{Font}
There are tons of fonts you could use:
\url{www.tug.dk/FontCatalogue}
\\
\begin{lstlisting}
% Change the font family to sans serif
	\renewcommand{\familydefault}{\sfdefault}
\end{lstlisting}


\subsection{Font Face}
\hspace*{4mm}\textbackslash emph			\hspace*{8mm}\emph{Text}
\\\hspace*{4mm}\textbackslash textbf 		\hspace*{8mm}\textbf{Text}
\\\hspace*{4mm}\textbackslash texttt 		\hspace*{9mm}\texttt{Text}
\\\hspace*{4mm}\textbackslash textrm 		\hspace*{7mm}\textrm{Text}
\\\hspace*{4mm}\textbackslash textsf 		\hspace*{9mm}\textsf{Text}
\\\hspace*{4mm}\textbackslash textsc 		\hspace*{8mm}\textsc{Text}
\\
\subsection{Font Size}
\hspace*{4mm}\textbackslash tiny			\hspace*{20mm}{\tiny{Text}}
\\\hspace*{4mm}\textbackslash scriptsize		\hspace*{11mm}{\scriptsize{Text}}
\\\hspace*{4mm}\textbackslash footnotesize	\hspace*{6mm}{\footnotesize{Text}}
\\\hspace*{4mm}\textbackslash small			\hspace*{18mm}{\small{Text}}
\\\hspace*{4mm}\textbackslash normalsize	\hspace*{8mm}{\normalsize{Text}}
\\\hspace*{4mm}\textbackslash large			\hspace*{18mm}{\large{Text}}
\\\hspace*{4mm}\textbackslash Large		\hspace*{16mm}{\Large{Text}}
\\\hspace*{4mm}\textbackslash LARGE		\hspace*{11mm}{\LARGE{Text}}
\\\hspace*{4mm}\textbackslash huge			\hspace*{17mm}{\huge{Text}}
\\\hspace*{4mm}\textbackslash Huge			\hspace*{16mm}{\Huge{Text}}
\\
\\
\rule{\linewidth}{0.1mm}




\subsection{Font Color}
For the font color, could check \url{https://www.rapidtables.com/web/color/RGB_Color.html}
\\
\begin{lstlisting}
% Add the package to change font color
	\usepackage{xcolor}
% Define customized colors
	\definecolor{comment}{RGB}{120,120,120}
	\definecolor{codenumber}{RGB}{100,100,100}
	\definecolor{shade}{RGB}{240,240,240}

% Change the font color
	\textcolor{color}{Text}
	{\color{color} Text}

\end{lstlisting}
~\\
\rule{\linewidth}{0.1mm}


\subsection{Font Background Color}
\begin{lstlisting}
% Add the package to change font background color
	\usepackage{xcolor}

% Change the font background color
	\colorbox{color}{Text}

\end{lstlisting}
~\\
\rule{\linewidth}{0.1mm}







% Hyperlinks
\section{Hyperlink}
To use hyperlink, you need to add the package:
\\\colorbox{shade}{\textbf{\textbackslash usepackage\{hyperref\}}}
\\All cross-referenced elements become hyperlinked.
\\Also could check \url{https://www.overleaf.com/learn/latex/Hyperlinks}


\subsection{Style \& Color}
\begin{lstlisting}
% Example of setting up the style and color
\hypersetup{
    colorlinks=true,
    linkcolor=blue,
    filecolor=magenta,      
    urlcolor=cyan,
}

% To set up the url font
% same: same with the main text
% rm: The font \rmfamily is used.
% sf: The font \sffamily is used.
% tt: This is the default: \ttfamily.
	\urlstyle{sf}
\end{lstlisting}
~\\





% Inserting picture
\section{Inserting Pictures}
Make sure to add the packages:
\\
\\\colorbox{shade}{\textbf{\textbackslash usepackage\{graphicx\}}} 
\\\colorbox{shade}{\textbf{\textbackslash usepackage\{wrapfig\}}} 
\\
\\
\rule{\linewidth}{0.1mm}


% Dummy example
\subsection{Dummy Example}
\begin{figure}[H]
\centering
% Include picture and define width (height automatically scales)
% Also scale=0.5 (Scales by half)
% angle=90 (Rotate 90 degrees)
\includegraphics[width=5cm]{example-grid-100x100pt}
\end{figure}
~\\
\\


\begin{lstlisting}
% h: Here,t: Topofpage,b: Bottom,p: SeparatePage
\begin{figure}[ht]
% make it centered
\centering
% add the picture, width could scale the image, default is \textwidth
\includegraphics[width=8cm]{example-grid-100x100pt}
\end{figure}
\end{lstlisting}

~\\
\rule{\linewidth}{0.1mm}

% wrapping image section
\subsection{Wrapping Image Example}
\begingroup

% Set space above and below float to 0
\setlength{\intextsep}{0pt}

% Set distance between columns
\setlength{\columnsep}{15pt}

% Text wraps around images
% Position image to the right with r
% Also can use l : Left, i : Inside Edge and o : Outside Edge
% 0.45\textwidth : Size image width relative to the text width
\begin{wrapfigure}{r}{0.45\textwidth}
\centering
% Place image
  \includegraphics[width=\linewidth]{example-image-duck}
  % I can assign a label I can refer to later
  \caption{Pretty Picture}\label{fig:prettypic}
\end{wrapfigure}
\blindtext

\endgroup

~\\
\begin{minipage}{\linewidth}
\begin{lstlisting}
% Group the text and image
\begingroup
% Set space above and below float to 0
\setlength{\intextsep}{0pt}

% Set distance between columns
\setlength{\columnsep}{15pt}

% Text wraps around images
% Position image to the right with r
% Also can use l : Left, i : Inside Edge and o : Outside Edge
% 0.45\textwidth : Size image width relative to the text width
\begin{wrapfigure}{r}{0.45\textwidth}
\centering
	% Place image
	\includegraphics[width=\linewidth]{pic.png}
	% I can assign a label I can refer to later
	\caption{Pretty Picture}\label{fig:prettypic}
\end{wrapfigure}

\blindtext

\endgroup
\end{lstlisting}
\end{minipage}
~\\
\\
\rule{\linewidth}{0.1mm}






% List section
\section{Lists}
\subsection{Bulleted Lists}\hspace*{\fill}
\\
\hspace*{6mm}{\large Smoothie Recipe}
\begin{itemize}[leftmargin=+.7in]
	\item 1 Cup Spinach
	\item 1 Cup Frozen Blueberries
	\item 2 Bananas
	\item 1.5 Cups Almond Milk
	\item Powders
	% Create a list in a list
	\begin{itemize}
		\item 1 Tbs PB2
		\item 1 Tsp Ambla Powder
	\end{itemize}
	\item 6 Dates
\end{itemize}
~\\
\begin{minipage}{\linewidth}
\begin{lstlisting}
% The title
\hspace*{6mm}{\large Smoothie Recipe}
% The list
\begin{itemize}
	\item 1 Cup Spinach
	\item 1 Cup Frozen Blueberries
	\item 2 Bananas
	\item 1.5 Cups Almond Milk
	\item Powders
		% Create a list in a list
		\begin{itemize}
			\item 1 Tbs PB2
			\item 1 Tsp Ambla Powder
		\end{itemize}
	\item 6 Dates
\end{itemize}
\end{lstlisting}
\end{minipage}
~\\
\rule{\linewidth}{0.1mm}



% Numbered lists section
\subsection{Numbered Lists}\hspace*{\fill}
\begin{lstlisting}
% Allows for any numbering scheme.
% Normally, nesting numbering goes from 1 to a to i. to A..
	\item[customNumbering]
\end{lstlisting}
~\\
\begin{lstlisting}
% Compact the list by replacing enumerate with compactenum 
% and itemize with com- pactitem.
	\usepackage{paralist}
\end{lstlisting}
~\\
\rule{\linewidth}{0.1mm}
\\
\\
\hspace*{6mm}{\large Perfect Meal Recipe}
\begin{enumerate}[label=\Roman*, font=\bfseries, leftmargin=+.7in]
	\item Add the following and cook for 2 minutes
	\begin{itemize}
		\item 1 tsp Olive Oil
		\item 1 Cup Onion, diced
		\item 3 cloves Garlic, minced
		\item 1 tsp Salt
		\item 1 Cup chopped Portobello Mushrooms
	\end{itemize}
	\item Add the following and stir for 2 minutes
	\begin{itemize}
		\item 2 TBs Curry Powder
		\item 1 tsp Fresh Minced Ginger
		\item 2 TBs Tomato Paste
	\end{itemize}
	\item Add the following and simmer for 15 minutes
	\begin{itemize}
		\item 1 cup uncooked Lentils
		\item 4 cups Vegetable Broth
	\end{itemize}
	\item Add the following and simmer for 20 minutes
	\begin{itemize}
		\item 2 cups chopped Carrots
		\item 4 Cups cubed Yams
	\end{itemize}
	\item Add the following and cook for 10 minutes
	\begin{itemize}
		\item 2 cups boiled diced Collard Greens
		\item 1 cup frozen diced Spinach
	\end{itemize}
\end{enumerate}
\leavevmode
\\
\begin{minipage}{\linewidth}
\begin{lstlisting}
% The title
\hspace*{6mm}{\large Perfect Meal Recipe}
% Can change the numbering \arabic*, \alpha*, \Alph*, \roman*
% Can change the font just for the numbering
\begin{enumerate}[label=\Roman*, font=\bfseries]
	\item Add the following and cook for 2 minutes
		\begin{itemize}
			\item 1 tsp Olive Oil
			\item 1 Cup Onion, diced
			\item 3 cloves Garlic, minced
			\item 1 tsp Salt
			\item 1 Cup chopped Portobello Mushrooms
		\end{itemize}
	\item Add the following and stir for 2 minutes
		\begin{itemize}
			\item 2 TBs Curry Powder
			\item 1 tsp Fresh Minced Ginger
			\item 2 TBs Tomato Paste
		\end{itemize}
	\item Add the following and simmer for 15 minutes
		\begin{itemize}
			\item 1 cup uncooked Lentils
			\item 4 cups Vegetable Broth
		\end{itemize}
	\item Add the following and simmer for 20 minutes
		\begin{itemize}
			\item 2 cups chopped Carrots
			\item 4 Cups cubed Yams
		\end{itemize}
	\item Add the following and cook for 10 minutes
		\begin{itemize}
			\item 2 cups boiled diced Collard Greens
			\item 1 cup frozen diced Spinach
		\end{itemize}
\end{enumerate}
\end{lstlisting}
\end{minipage}
~\\
\\
\rule{\linewidth}{0.1mm}


% Definition lists
\subsection{Definition List}\hspace*{\fill}
\\
\begin{description}
	\item[Philtrum] The vertical groove on the median line of the upper lip
	\item[Darkle] Becoming cloudy or dark
	\item[Pogonotrophy] Growing and grooming a beard or other facial hair
	\item[Interrobang] A punctuation mark designed for use especially at the end of an exclamatory rhetorical question; usually written as ?!
\end{description}
~\\
\begin{minipage}{\linewidth}
\begin{lstlisting}
\begin{description}
	\item[Philtrum] The vertical groove on the median line of the upper lip
	\item[Darkle] Becoming cloudy or dark
	\item[Pogonotrophy] Growing and grooming a beard or other facial hair
	\item[Interrobang] A punctuation mark designed for use especially at 
		the end of an exclamatory rhetorical question; usually written as ?!
\end{description}
\end{lstlisting}
\end{minipage}
~\\
\rule{\linewidth}{0.1mm}



% Table section
\section{Tables}
\subsection{E.g. tabbing}
\begin{tabbing}

% On setup row define where tabs occur and the space to set aside
Customer  \= Name \hspace*{1.5cm} \= Street \hspace*{1.5cm} \= City \\

% \> Jumps to the next tab
\> Derek Banas \> 123 Main St \> Pittsburgh \\
\end{tabbing}

\begin{lstlisting}  
\begin{tabbing}

% On setup row define where tabs occur and the space to set aside
Customer  \= Name \hspace*{1.5cm} \= Street \hspace*{1.5cm} \= City \\

% \> Jumps to the next tab
\> Echo \> 123 Main St \> Toronto \\
\end{tabbing}
\end{lstlisting}  
~\\
\rule{\linewidth}{0.1mm}



\subsection{E.g. Tables, Type Emphasis \& Fonts}
Below is an example of a more complex table:
\\To fix the floating problem (the table will apear before the title), use \textbf{\textbackslash usepackage\{float\}}, and add the \textbf{[H]} parameter after \textbf{\textbackslash begin\{table\}}.
\\
\begin{table}[H]
\center
	\begin{tabular}{l|l|l}
	\hline
	\textbf{Name} & \textbf{Command} & \textbf{Sample Text} \\
	\hline
	
	emphasize & \verb|\emph| & \emph{abcdefgh} \\
	italic & \verb|\textit| & \textit{abcdefgh} \\
	slanted & \verb|\textbf| & \textbf{abcdefgh} \\
	bold & \verb|\emph| & \emph{abcdefgh} \\
	small capped & \verb|\textsc| & \textsc{abcdefgh} \\
	medium & \verb|\textmd| & \textmd{abcdefgh} \\
	upright & \verb|\textup| & \textup{abcdefgh} \\
	roman family & \verb|\textrm| & \textrm{abcdefgh} \\
	sans serif & \verb|\textsf| & \textsf{abcdefgh} \\
	typewriter & \verb|\texttt| & \texttt{abcdefgh} \\
	combo & \verb|\textup{\textbf{}}| & \textit{\textbf{abcdefgh}} \\
	\end{tabular}
	\caption{Ways to emphasize text}
\end{table}
~\\

\begin{lstlisting}  
% Wrap a table around tabular to make captions
% Use the [H] parameter to avoid floating
\begin{table}[H]
\center
	% Define a table with 3 left aligned columns 
	% Use c : Center) or r : Right
	% The center | creates a vertical line
	\begin{tabular}{l|l|l}
	% Draw horizontal line
	% \usepackage{booktabs} provides different line thicknesses
	% \toprule, \midrule, \bottomrule
	\hline
	
	% '&' Defines the breaks in the table
	\textbf{Name} & \textbf{Command} & \textbf{Sample Text} \\
	\hline
	
	% \verb| | allows you to type commands
	emphasize & \verb|\emph| & \emph{abcdefgh} \\
	italic & \verb|\textit| & \textit{abcdefgh} \\
	slanted & \verb|\textbf| & \textbf{abcdefgh} \\
	bold & \verb|\emph| & \emph{abcdefgh} \\
	small capped & \verb|\textsc| & \textsc{abcdefgh} \\
	medium & \verb|\textmd| & \textmd{abcdefgh} \\
	upright & \verb|\textup| & \textup{abcdefgh} \\
	roman family & \verb|\textrm| & \textrm{abcdefgh} \\
	sans serif & \verb|\textsf| & \textsf{abcdefgh} \\
	typewriter & \verb|\texttt| & \texttt{abcdefgh} \\
	combo & \verb|\textup{\textbf{}}| & \textit{\textbf{abcdefgh}} \\
	\end{tabular}
	\caption{Ways to emphasize text}
\end{table}
\end{lstlisting}  
~\\
\rule{\linewidth}{0.1mm}




\subsection{E.g. Leave some columns blank \& merge others}
An example of leaving some columns blank and merge others:
\\
\begin{table}[H]
\center
\begin{tabular}{@{}*3l@{}}
\multicolumn{2}{c}{Name} &
\multicolumn{1}{c}{Age}\\
First & Last & \\
\hline
Derek & Banas & 44\\
Sally & Smith & 42\\
\end{tabular}
\end{table}
~\\

\begin{minipage}{\linewidth}
\begin{lstlisting}  
% Use the [H] parameter to avoid floating
\begin{table}[H]
\center
% '@{} ' defines the space before the column
% having nothing between the {} means I want no space
% Create 3 columns
\begin{tabular}{@{}*3l@{}}

% Take up to columns with Name and 1 with age
\multicolumn{2}{c}{Name} &
\multicolumn{1}{c}{Age}\\

% Add headers with the last left blank
First & Last & \\
\hline
Derek & Banas & 44\\
Sally & Smith & 42\\
\end{tabular}
\end{table}

\end{lstlisting}
\end{minipage}
~\\
\\
\rule{\linewidth}{0.1mm}




\section{Accent Characters}
These is a package that allows you to enter accented characters:
\\
\textbf{\textbackslash usepackage[utf8]\{inputenc\}} on Linux/MacOS and
\\
\textbf{\textbackslash usepackage[latin1]\{inputenc\}} on Windows.
\\
\\
Some examples:
\\
\'{a} \^{e} `{o} \"{u} \.{a} \={o} \~{n} \u{a} \H{a} \v{e} \t{oo} \c{c} \d{n} \b{i}




\section{Type Emphasis}
\subsection{Simple Example}
\begin{lstlisting} 
% Abbreviate listing in table of contents to just Type and make sans serif font
% I use label here so I can refer to this section later
\section[Type]{\textsf{Type Emphasis \& Sizing}} \label{sec:typeemp}
\end{lstlisting}  
~\\
Result:
\begin{figure}[ht]
\includegraphics[width=6cm]{emphasis.png}
\end{figure}
~\\
\rule{\linewidth}{0.1mm}
\subsection{Quote}
\begin{quote}
``I like long walks, especially when they are taken by people who annoy me.'' 
- Fred Allen
\end{quote} 
\begin{lstlisting}  
% Create quotes
\begin{quote}
"I like long walks, especially when they are taken by people who annoy me.'' 
- Fred Allen
\end{quote}
\end{lstlisting}  
~\\
\rule{\linewidth}{0.1mm}





\section{Math Formulas}
Add the package \textbf{\textbackslash usepackage\{amsmath\}}
\\
List of LaTeX Symbols \url{http://www.rpi.edu/dept/arc/training/latex/LaTeX_symbols.pdf}



\subsection{E.g. Quadratic Equation}
\begin{flalign*}
	& ax^2 + bx + c = 0 &\\
\end{flalign*}

\begin{lstlisting}  
% Quadratic Equation
% flalign* with formula surrounded with &s makes it left justified
\begin{flalign*}
	& ax^2 + bx + c = 0 &\\
\end{flalign*}
\end{lstlisting}  
~\\
\rule{\linewidth}{0.1mm}


\subsubsection{Place a Formula in Text}
This \( ax^2 + bx + c = 0 \) is the quadratic equation
\\
\begin{lstlisting}  
This \( ax^2 + bx + c = 0 \) is the quadratic equation
\end{lstlisting}  
~\\
\rule{\linewidth}{0.1mm}






\subsubsection{\$ Tex Shortcut}
Formula: $x=\frac{-b\pm\sqrt{b^2-4ac}}{2a}$
\\
\begin{lstlisting}  
Formula: $x=\frac{-b\pm\sqrt{b^2-4ac}}{2a}$
\end{lstlisting}  
~\\
\rule{\linewidth}{0.1mm}





\subsection{Matrices}
~\\
$\begin{pmatrix} 
1 & 2 \\ 
3 & 4 
\end{pmatrix}$
\\
\\
\begin{lstlisting}  
$\begin{pmatrix} 
1 & 2 \\ 
3 & 4 
\end{pmatrix}$
\end{lstlisting}  
~\\
\rule{\linewidth}{0.1mm}




\subsection{Integrals}
~\\
$\Delta x=\int_{t_0}^{t_1} v(t)dt$
\\
\begin{lstlisting}  
$\Delta x=\int_{t_0}^{t_1} v(t)dt$
\end{lstlisting}  
~\\
\rule{\linewidth}{0.1mm}




\subsection{Limits}
~\\
$\lim_{x\to0} \frac 1 x = \infty$
\\
\begin{lstlisting}  
$\lim_{x\to0} \frac 1 x = \infty$
\end{lstlisting}  
~\\
\rule{\linewidth}{0.1mm}




\subsection{Summations}
~\\
$e^x=\sum_{n=0}^\infty\frac{x^n}{n!}$
\\
\begin{lstlisting}  
$e^x=\sum_{n=0}^\infty\frac{x^n}{n!}$
\end{lstlisting}  
~\\
\rule{\linewidth}{0.1mm}


\subsection{Some Commonly Used Symbols}
\begin{table}[H]
\center
	\begin{tabular}{l|l}
	\hline
	\textbf{Command} & \textbf{Sample Text} \\
	\hline
	\hline
	
	\verb|\alpha| & $\alpha$ \\
	\verb|\beta| & $\beta$ \\
	\verb|\gamma| & $\gamma$ \\
	\verb|\delta| & $\delta$ \\
	\verb|\epsilon| & $\epsilon$  \\
	\verb|\zeta| &  $\zeta$\\
	\verb|\theta| &  $\theta$\\
	\verb|\vartheta| & $\vartheta$\\
	\verb|\lambda| &  $\lambda$\\
	\verb|\Lambda| & $\Lambda$\\
	\verb|\mu| & $\mu$\\
	\verb|\nu| & $\nu$\\
	\verb|\xi| & $\xi$\\
	\verb|\Xi| & $\Xi$\\
	\verb|\pi| & $\pi$\\
	\verb|\Pi | & $\Pi$  \\
	\verb|\rho | & $\rho$  \\
	\verb|\sigma | & $\sigma$  \\
	\verb|\Sigma | & $\Sigma$  \\
	\verb|\tau | & $\tau$  \\
	\verb|\upsilon | & $\upsilon$  \\
	\verb|\Upsilon | & $\Upsilon$  \\
	\verb|\phi | & $\phi$  \\
	\verb|\varphi | & $\varphi$  \\
	\verb|\Phi | & $\Phi$  \\
	\verb|\psi | & $\psi$  \\
	\verb|\Psi | & $\Psi$  \\
	\verb|\Omega | & $\Omega$  \\
	\verb|\omega | & $\omega$  \\
	\end{tabular}
	\caption{Greek letters examples}\label{sec:greek}
\end{table}


\begin{table}[H]
\center
	\begin{tabular}{l|l}
	\hline
	\textbf{Command} & \textbf{Sample Text} \\
	\hline
	\hline
	
	\verb|\mathcal{A}| & $\mathcal{A}$ \\
	\verb|\mathcal{B}| & $\mathcal{B}$ \\
	\end{tabular}
	\caption{Script letters examples}
\end{table}

\begin{table}[H]
\center
	\begin{tabular}{l|l}
	\hline
	\textbf{Command} & \textbf{Sample Text} \\
	\hline
	\hline
	
	\verb|t_0| & $t_0$ \\
	\end{tabular}
	\caption{Subscript examples}
\end{table}


\begin{table}[H]
\center
	\begin{tabular}{l|l}
	\hline
	\textbf{Command} & \textbf{Sample Text} \\
	\hline
	\hline
	
	\verb| x^2 | & $ x^2 $ \\
	\end{tabular}
	\caption{Superscript examples}
\end{table}


\begin{table}[H]
\center
	\begin{tabular}{l|l}
	\hline
	\textbf{Command} & \textbf{Sample Text} \\
	\hline
	\hline
	
	\verb| \vec{a}\cdot\hat{x}=a_x | & $ \vec{a}\cdot\hat{x}=a_x $ \\
	\end{tabular}
	\caption{Vectors examples}
\end{table}



\begin{table}[H]
\center
	\begin{tabular}{l|l}
	\hline
	\textbf{Command} & \textbf{Sample Text} \\
	\hline
	\hline
	
	\verb|\arccos| & $\arccos$ \\
	\verb|\arcsin| & $\arcsin$ \\
	\verb|\arctan| & $\arctan$ \\
	\verb|\arg| & $\arg$ \\
	\verb|\cos| & $\cos$  \\
	\verb|\cosh| &  $\cosh$\\
	\verb|\cot| &  $\cot$\\
	\verb|\coth| & $\coth$\\
	\verb|\exp| &  $\exp$\\
	\verb|\lim| & $\lim$\\
	\verb|\lg| & $\lg$\\
	\verb|\ln| & $\ln$\\
	\verb|\max| & $\max$\\
	\verb|\min| & $\min$\\
	\verb|\sin| & $\sin$\\
	\verb|\sinh | & $\sinh$  \\
	\verb|\tan | & $\tan$  \\
	\verb|\tanh | & $\tanh$  \\

	\end{tabular}
	\caption{Operators examples}
\end{table}




\begin{table}[H]
\center
	\begin{tabular}{l|l}
	\hline
	\textbf{Command} & \textbf{Sample Text} \\
	\hline
	\hline
	
	\verb|\leftarrow| & $\leftarrow$ \\
	\verb|\Leftarrow| & $\Leftarrow$ \\
	\verb|\rightarrow| & $\rightarrow$ \\
	\verb|\Rightarrow| & $\Rightarrow$ \\
	\verb|\leftrightarrow| & $\leftrightarrow$  \\
	\verb|\rightleftharpoons| &  $\rightleftharpoons$\\
	\verb|\uparrow| &  $\uparrow$\\
	\verb|\downarrow| & $\downarrow$\\
	\verb|\Uparrow| &  $\Uparrow$\\
	\verb|\Downarrow| & $\Downarrow$\\
	\verb|\Leftrightarrow| & $\Leftrightarrow$\\
	\verb|\Updownarrow| & $\Updownarrow$\\
	\verb|\mapsto| & $\mapsto$\\
	\verb|\longmapsto| & $\longmapsto$\\
	\verb|\nearrow| & $\nearrow$\\
	\verb|\searrow | & $\searrow$  \\
	\verb|\swarrow | & $\swarrow$  \\
	\verb|\nwarrow | & $\nwarrow$  \\
	\verb|\leftharpoonup | & $\leftharpoonup$  \\
	\verb|\rightharpoonup | & $\rightharpoonup$  \\
	\verb|\leftharpoondown | & $\leftharpoondown$  \\
	\verb|\rightharpoondown | & $\rightharpoondown$  \\
	
	\end{tabular}
	\caption{Arrows examples}
\end{table}




\begin{table}[H]
\center
	\begin{tabular}{l|l}
	\hline
	\textbf{Command} & \textbf{Sample Text} \\
	\hline
	\hline
	
	\verb|\geq| & $\geq$ \\
	\verb|\gg| & $\gg$ \\
	\verb|\leq| & $\leq$ \\
	\verb|\ll| & $\ll$ \\
	\verb|\neq| & $\neq$  \\
	
	\end{tabular}
	\caption{Relational Operators examples}
\end{table}



\begin{table}[H]
\center
	\begin{tabular}{l|l}
	\hline
	\textbf{Command} & \textbf{Sample Text} \\
	\hline
	\hline
	
	\verb|\approx| & $\approx$ \\
	\verb|\asymp| & $\asymp$ \\
	\verb|\bowtie| & $\bowtie$ \\
	\verb|\cong| & $\cong$ \\
	\verb|\dashv| & $\dashv$  \\
	\verb|\doteq| &  $\doteq$\\
	\verb|\equiv| &  $\equiv$\\
	\verb|\frown| & $\frown$\\
	\verb|\mid| &  $\mid$\\
	\verb|\models| & $\models$\\
	\verb|\parallel| & $\parallel$\\
	\verb|\perp| & $\perp$\\
	\verb|\prec| & $\prec$\\
	\verb|\preceq| & $\preceq$\\
	\verb|\propto| & $\propto$\\
	\verb|\sim | & $\sim$  \\
	\verb|\simeq | & $\simeq$  \\
	\verb|\smile | & $\smile$  \\
	\verb|\succ | & $\succ$  \\
	\verb|\succeq | & $\succeq$  \\
	\verb|\vdash | & $\vdash$  \\

	\end{tabular}
	\caption{Binary Operation/Relation Symbols examples}
\end{table}
~\\
\rule{\linewidth}{0.1mm}


\section{Custom Commands}
We could define our own commands by the form : 
\\
\textbf{\textbackslash newcommand\{Command Name\}\{Content\}}
\\
But make sure to put it in the beginning.

\begin{lstlisting}  
% Will show 'New Think Tank' if type \NTT\
	\newcommand{\NTT}{New Think Tank}
	
%  Will show 'New Think Tank' in bold if type \NTTB\
	\newcommand{\NTT}{\textbf{New Think Tank}}

% Will apply the font to the text if type '\typew{typewriter} '
	\newcommand{\typew|[1]{\texttt{#1}}
\end{lstlisting}  
~\\
 \rule{\linewidth}{0.1mm}
 
 
 
 
 
 \section{Text Boxes \& Justification}
 \subsection{Centering the text}
 ~\\
 {\centering
Sample Text Sample Text Sample Text Sample Text\\
}

\begin{lstlisting}  
{\centering
Sample Text Sample Text Sample Text Sample Text\\
}
\end{lstlisting}  
~\\
\rule{\linewidth}{0.1mm}

\subsection{Justified Column}
~\\
\quad\parbox{2cm}{I used to think I was indecisive, but now I'm not too sure.}
\quad\parbox{2cm}{Always re\-mem\-ber that you're unique. Just like everyone.}
\quad\parbox{2cm}{\raggedright I always wanted to be somebody, but I should have been more specific.}
\quad\parbox{2cm}{\raggedleft When I was a kid my parents moved a lot, but I always found them. }
~\\
\\
\\
\begin{minipage}{\linewidth}
\begin{lstlisting}  
% Create a fully justified column 4cm wide
% \parbox{4cm}{I used to think I was indecisive, but now I'm not too sure.}
% [t]{2cm} aligns to top, [b] aligns to bottom, [s] stretches vertically, [c] centers
% \quad adds horizontal spacing between the boxes
\quad\parbox{2cm}{I used to think I was indecisive, but now I'm not too sure.}
% You can define hyphenation with\-
\quad\parbox{2cm}{Always re\-mem\-ber that you're unique. Just like everyone.}
% raggedright eliminates justification and hyphenation (Justifies Left)
% You can do this document wide if typed in the preamble
\quad\parbox{2cm}{\raggedright I always wanted to be somebody, but I should have been more specific.}
% raggedleft eliminates justification and hyphenation (Justifies Right)
\quad\parbox{2cm}{\raggedleft When I was a kid my parents moved a lot, but I always found them. }
\end{lstlisting}  
\end{minipage}
~\\
\rule{\linewidth}{0.1mm}







\section{Referencing}
\subsection{Footer Referencing}
The answer you're looking for is inside of you, but it's wrong.\footnote[2]{author unknown} 
\\
\begin{lstlisting}  
% You can add footnotes and override the numbering 
The answer you're looking for is inside of you, but it's wrong.\footnote[2]{author unknown}
\end{lstlisting}  
~\\
\rule{\linewidth}{0.1mm}




\subsection{Table Referencing}
There is a great table on Greek letter symbols is in section~\ref{sec:greek} on page~\pageref{sec:greek}
\\
\begin{lstlisting}  
There is a great table on Type Emphasis is in this section~\ref{sec:typeemp} on page~\pageref{sec:typeemp}
\end{lstlisting}  
~\\
\rule{\linewidth}{0.1mm}

\subsection{Image Referencing}
There is a pretty picture in section~\ref{fig:prettypic} on page~\pageref{fig:prettypic}\\[2pt]
\begin{lstlisting}  
There is a pretty picture in section~\ref{fig:prettypic} on page~\pageref{fig:prettypic}\end{lstlisting}  
~\\
\rule{\linewidth}{0.1mm}



\subsection{Bibliograph}
~\\
How I learned my ABCs \cite{ABCAFR}.
\\
\begin{thebibliography} {books}
\bibitem{ABCAFR} Walter Abish \emph{The Alphabetical Africa}, 1974
\end{thebibliography}
~\\
\begin{lstlisting}  
% Listing references in a bibliography
How I learned my ABCs \cite{ABCAFR}. 

% Generate the bibliograph elsewhere in this document
\begin{thebibliography} {books}
\bibitem{ABCAFR} Walter Abish \emph{The Alphabetical Africa}, 1974
\end{thebibliography}
\end{lstlisting}  
~\\
\rule{\linewidth}{0.1mm}








\subsection{Index Referencing}
Add the package: 
\\
\textbf{\textbackslash usepackage\{index\}}
\\
\textbf{\textbackslash makeindex}
\\
Remember to choose 'Makeindex' under Typeset.
\\
\\
\\
\\
Example:
\\
\\
When I was born I was so ugly the doctor slapped my mother - {\index{Rodney}Rodney Dangerfield}
\\
\blindtext[1]
\addcontentsline{toc}{chapter}{Index}
\printindex
~\\
\begin{lstlisting}  
When I was born I was so ugly the doctor slapped my mother - {\index{Rodney}Rodney Dangerfield}
\\
\blindtext[1]
\clearpage
\addcontentsline{toc}{chapter}{Index}
\printindex
\end{lstlisting}  



\end{document}



