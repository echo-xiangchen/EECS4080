\documentclass[a4paper,10pt,titlepage]{article}



\usepackage{index}
\makeindex
\usepackage[a4paper, inner=2cm, outer=2cm, top=2cm, bottom=2cm, bindingoffset=0cm]{geometry}
\linespread{0.5}
\usepackage[english]{babel}

% define my own color
\usepackage{xcolor}
\definecolor{comment}{RGB}{120,120,120}
\definecolor{codenumber}{RGB}{100,100,100}
\definecolor{shade}{RGB}{220,220,220}
\definecolor{link}{RGB}{0,0,225}
\definecolor{url}{RGB}{30,144,255}
\definecolor{jackie}{RGB}{11,23,70}

\usepackage{framed}
\usepackage{graphicx}
\usepackage{wrapfig}
\usepackage{blindtext}

%make the section title centered
\usepackage{sectsty}
\sectionfont{\centering}

\usepackage{enumitem}


% use font package
\usepackage{fontspec}
% change the font
\setmonofont{Menlo}


% provides different line thicknesses for tables
\usepackage{booktabs}

%prevents figures ever floating backwards up the current page
\usepackage{flafter}

% here for H placement parameter
\usepackage{float} 

\usepackage{makecell, caption}

% enable inserting codes
\usepackage{listings}

\lstset{
  language={Java},
  commentstyle=\color{comment},
  basicstyle=\footnotesize\ttfamily,
  numbers=left,
  numberstyle=\scriptsize\color{codenumber},
  frame=single,		% add the frame
  breaklines=true,	% sets automatic line breaking
 % texcsstyle=*\color{red},
 % identifierstyle=\color{magenta},
  extendedchars,
  escapeinside=``,	% used for escaping
  tabsize=2
}

\lstset{
  literate=*{¡\\!}{{\textcolor{red}{\textbackslash!}}}{2}
            {¡\\"}{{\textcolor{red}{\textbackslash"}}}{2}
            {¡\\\#}{{\textcolor{red}{\textbackslash\#}}}{2}
            {¡\\\$}{{\textcolor{red}{\textbackslash\$}}}{2}
            {¡\\\%}{{\textcolor{red}{\textbackslash\%}}}{2}
            {¡\\\&}{{\textcolor{red}{\textbackslash\&}}}{2}
            {¡\\'}{{\textcolor{red}{\textbackslash'}}}{2}
            {¡\\(}{{\textcolor{red}{\textbackslash(}}}{2}
            {¡\\)}{{\textcolor{red}{\textbackslash)}}}{2}
            {¡\\*}{{\textcolor{red}{\textbackslash*}}}{2}
            {¡\\+}{{\textcolor{red}{\textbackslash+}}}{2}
            {¡\\,}{{\textcolor{red}{\textbackslash,}}}{2}
            {¡\\-}{{\textcolor{red}{\textbackslash-}}}{2}
            {¡\\.}{{\textcolor{red}{\textbackslash.}}}{2}
            {¡\\/}{{\textcolor{red}{\textbackslash/}}}{2}
            {¡\\:}{{\textcolor{red}{\textbackslash:}}}{2}
            {¡\\;}{{\textcolor{red}{\textbackslash;}}}{2}
            {¡\\<}{{\textcolor{red}{\textbackslash<}}}{2}
            {¡\\=}{{\textcolor{red}{\textbackslash=}}}{2}
            {¡\\>}{{\textcolor{red}{\textbackslash>}}}{2}
            {¡\\?}{{\textcolor{red}{\textbackslash?}}}{2}
            {¡\\[}{{\textcolor{red}{\textbackslash[}}}{2}
            {¡\\\\}{{\textcolor{red}{\textbackslash\textbackslash}}}{2}
            {¡\\]}{{\textcolor{red}{\textbackslash]}}}{2}
            {¡\\\^}{{\textcolor{red}{\textbackslash\textasciicircum}}}{2}
            {¡\\\{}{{\textcolor{red}{\textbackslash\{}}}{2}
            {¡\\|}{{\textcolor{red}{\textbackslash|}}}{2}
            {¡\\\}}{{\textcolor{red}{\textbackslash\}}}}{2}
            {¡\\\~}{{\textcolor{red}{\textbackslash\textasciitilde}}}{2}
}

% E.g.
% \begin{lstlisting}
% ¡\! ¡\" ¡\# ¡\$ ¡\% ¡\& ¡\' ¡\( ¡\) ¡\* ¡\+ ¡\, ¡\- ¡\. ¡\/ ¡\: ¡\; ¡\< ¡\= ¡\> ¡\? ¡\[ ¡\\ ¡\] ¡\^ ¡\{ ¡\| ¡\} ¡\~
% \end{lstlisting}



% Use hyperlink
\usepackage{hyperref}
\newcommand\fnurl[2]{%
  \href{#2}{#1}\footnote{\url{#2}}%
}

\urlstyle{sf}

%\hypersetup{
%    colorlinks=true,
%    linkcolor=jackie,
%    filecolor=magenta,      
%    urlcolor=jackie,
%}


% For math formulas
\usepackage{amsmath}




% documents begins here
\begin{document}
\centerline {\textbf {\LARGE Xiang Chen}}
\centerline {818-320 Assiniboine Road, Toronto, ON M3J 1L1}
\centerline {(647) 781 6129 \hspace*{1mm}|\hspace*{1mm} xiang2@my.yorku.ca}
\leftline{\rule{\linewidth}{0.1mm}}
~\\
\textbf {\Large Education}
\begin{itemize}%[topsep=3pt]
%\setlength{\itemsep}{-1pt}
	\item {\bf Master of Mathematics, Computer Science}
	\\\emph {University of Waterloo, Waterloo, Canada} {\hfill \emph{Expected Sep 2020 - Aug 2022}}
	\item {\bf Honors Bachelor of Science, Computer Science}     - GPA: 3.8/4.0
	\\\emph {York University, Toronto, Canada} {\hfill \emph{2017 - 2020}}
	\item {\bf Ontario College Advanced Diploma, Computer Systems Technology}     - GPA: 4.0/4.0
	\\\emph {Seneca College, Toronto, Canada}{\hfill \emph{2015 - 2016}}
	\item {\bf Honors Bachelor of Economics, International Economics and Trade}
	\\\emph  {Guangzhou University of Chinese Medicine, Guangzhou, China}{\hfill \emph{2008 - 2012}}
\end{itemize}

%\begin{tabbing}
%	\\\hspace*{2mm}{\bf Relevant Courses} \\
%	\hspace*{4mm}Software Design \hspace*{2cm} \= Web-based and Object-Oriented Programming  \\
%	\hspace*{4mm}Compilers and Interpreters \> Data Structures and Algorithms  \\
%	\hspace*{4mm}Artificial Intelligence \> Machine Learning \\
%	\end{tabbing}
\leftline{\rule{\linewidth}{0.1mm}}
~\\
\textbf {\Large Scholarship}
\begin{itemize}%[topsep=3pt]
%\setlength{\itemsep}{-1pt}
	\item Lassonde Undergraduate Research Award (LURA): \emph {\$8,000} {\hfill \emph {Apr 2020}}
	\item York University Continuing Student Scholarship: \emph {based on academic merit} {\hfill \emph {Sep 2018}}
\end{itemize}
\rule{\linewidth}{0.1mm}
~\\
\textbf {\Large Technical Skills} 
\begin{itemize}%[topsep=3pt]
%\setlength{\itemsep}{-1pt}
	\item {\bf Languages:} Java, Python, Prolog, C, Shell Programming, Javascript, \LaTeX, HTML, MatLab, Eiffel (Design by Contract),  Verilog
	%\item {\bf Markup Language:}  
	%\item {\bf Scripting Language:} 
	\item {\bf Tools:} Antlr 4, Z3 SMT Solver, Eclipse, SWI-Prolog, PyCharm
\end{itemize}
\rule{\linewidth}{0.1mm}
~\\
\textbf {\Large Projects}
% Building an Automated Verifier for a Procedural Programming Language
\begin{itemize}%[topsep=3pt]
%\setlength{\itemsep}{-1pt}
	\item {\bf Building an Automated Verifier for a Procedural Programming Language}
	\\\emph {Software Engineering Project, York University} {\hfill \emph{Jan 2020 - present}}
	\begin{itemize}%[topsep=1pt]
	%\setlength{\itemsep}{-1.5pt}
		\item Extended project of previous project that involves a series of extensions:
		\item Addition of linear data structures (e.g., array, pair, as well as their combinations).
		\item Extension to the proposition/predicate language to include constructs of a procedural programming language, including: variable assignments, sequential composition, conditionals, loops, routines, and contractual specification (preconditions, postconditions, and class invariants).
		\item The automated transformation of each routine into a {\bf Hoare Triple} that can either be proved as a tautology or disproved via counterexamples. The process of (automatically) proving the Hoare Triples also involves the systematic calculation of the {\bf weakest precondition} of every routine, given its implementation and postcondition.
		\item Project GitHub: \url{https://github.com/echo-xiangchen/EECS4080}
		\item Presentation Recording: \url{https://youtu.be/7RtcP-6Lffk}
\end{itemize}
\end{itemize}
% Building an Automated Verifier for Propositional and Predicate Logic
\begin{itemize}%[topsep=3pt]
%\setlength{\itemsep}{-1pt}
	\item {\bf Building an Automated Verifier for Propositional and Predicate Logic}
	\\\emph {Software Engineering Project, York University} {\hfill \emph{Aug 2019 - Dec 2019}}
	\begin{itemize}%[topsep=1pt]
	%\setlength{\itemsep}{-1.5pt}
		\item Use context-free grammar and the {\bf Antlr tool} to specify an expression language supporting the writing of propositions, predicates, and simple data structures (e.g., lists).
		\item Implement a series of transformation rules (as {\bf Java} methods using Visitor Design Pattern), which will return runtime abstract syntax tree (AST) based on any input text conforming to the language grammar.
		\item Turn the abstract syntax tree (AST) into the equivalent encoding in {\bf Z3 SMT Solver} for verification.
		\item Develop a complete set of tests, both at the unit level (i.e., individual transformation rules) and at the acceptance level (i.e., end-to-end verification).
		\item Users of this verification tool can easily check if a valid propositional or predicate formula is a tautology, or receive a counterexample otherwise.
		\item Project GitHub: \url{https://github.com/echo-xiangchen/EECS4080}
		\item Presentation Recording: \url{https://youtu.be/LHthLnmz6Bo}
	\end{itemize}
\end{itemize}
% Battleship Game Project
\begin{itemize}%[topsep=3pt]
%\setlength{\itemsep}{-1pt}
	\item {\bf Battleship Game Project}
	\\\emph {Eiffel ETF and Software Engineering Project, York University} {\hfill \emph{Jan 2019 - Apr 2019}}
	\begin{itemize}%[topsep=1pt]
	%\setlength{\itemsep}{-1pt}
		\item Use {\bf Eiffel} ETF framework to develop a text-based UI battleship game, including the features supporting normal game mode, debug game mode, and some extra operations (e.g., undo/redo, reset, give-up).
		\item Design the structure with information hiding principle, modularity, abstraction and separation of concerns.
		\item Develop a series of regression tests and a formal report including a clear design structure using BON diagram.
	\end{itemize}
\end{itemize}
\rule{\linewidth}{0.1mm}
~\\
\textbf {\Large Employment Experience}
% Personal Assistant
\begin{itemize}%[topsep=3pt]
%\setlength{\itemsep}{-1pt}
	\item {\bf Teaching Assistant} of course EECS 1022: Programming for Mobile Computing {\hfill \emph {Winter 19, Summer 19}}
	\begin{itemize}%[topsep=1pt]
	%\setlength{\itemsep}{-1pt}
		\item Lead TA sessions 3 hours a week, and grade lab assignments.
		\item Assist professor and help students with their labs (using Java, developing Android Apps).
		\item Deepen students' understanding of Java by explain simple concepts and algorithms to them.
	\end{itemize}
	\item {\bf Government Officer}
	\\\emph {Government of Yayao, Guangzhou, China} {\hfill \emph{Aug 2013 – Dec 2014}}
	\begin{itemize}%[topsep=1pt]
	%\setlength{\itemsep}{-1pt}
	\item Effectively manage the Collective Management Trading Platform through updating, monitoring and preserving data quality.
	\item Maintaine the hardware and software of office computers, which helps me generate the interest of computers.
	\item Involve in writing the web page of the Collective Management Trading Platform as well as a simple Windows program that could access and review the information on the platform.
	\end{itemize}
	% personal assisstant
	\item {\bf Assistant} of Dr. Nanshan, Zhong (Academician of Chinese	Academy of Engineering)
	\\\emph {Guangzhou Respiratory Institute, Guangzhou, China} {\hfill \emph{Aug 2012 – Jul 2013}}
	\begin{itemize}%[topsep=3pt]
%\setlength{\itemsep}{-1pt}
	\item Develop the proposal for the cooperation of Nanshan Medical Development Foundation and Mr. Zhenyu, Huo, and make meeting summaries.
	\item Involve in the research of Investigation on the physical health of Macao adolescents, attend the meeting in Macao Institution for Applied Research in Medicine and Health (MIAR), and generate the progress report as well as meeting summaries, also help drafting the final paper.
	\item Organize the annual meeting of Guangdong Medical Association in 2012, and create the  formal report.
	\end{itemize}
\end{itemize}


%\textbf {\Large Extra-Curricular Activities}
%\begin{itemize}
%	\item {\bf Member}
%	\\\emph {Entertainment Department of Student Union, Guangzhou University of Chinese Medicine} {\hfill \emph{Sep 2008 – Jul 2009}}
%	\item {\bf Vice Minister}
%	\\\emph {Entertainment Department of Student Union, Guangzhou University of Chinese Medicine} {\hfill \emph{Sep 2009 – Jul 2012}}
%\end{itemize}

\end{document}